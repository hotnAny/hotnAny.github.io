\subsection{Primary research projects}

\cvitem{
	2019
} {
	New Scientists. "Turn any object into a robot using this program and a 3D printer"
}

\cvitem{
	2019
} {
	ACM TechNews. "Turn any object into a robot using this program and a 3D printer"
}

\cvitem{
	2019
} {
	Hackster.io. "Robiot Is a Design Tool That Generates Mechanisms to Motorize Everyday Objects"
}

\cvitem{
	2019
} {
Innovation Cloud. "Innovation that will turn everyday objects into robots"
}

\cvitem{
	2019
} {
	Fabbaloo. "Robiot Can Automatically Design Handy Household Machines"
}


\cvitem{
	2018
} {
	3ders.org. "Forté: user-driven generative design tool for easy optimization of 3D printed objects"
}

\cvitem{
	2018
} {
	All3DP. "Forté Lets you Draw in 2D, Creates 3D Generative Designs Automatically"
}

\cvitem{
	2018
} {
	3DShoes.com. "Forté Design Tool"
}

\cvitem{
	2018
} {
	FutureLab3D. "Forte: user-driven generative design tool for easy optimization of 3D printed objects"
}

\cvitem{
	2018
} {
	3D Adept. "Forte, the generative design tool that will ease the optimization of 3D printed objects"
}

\cvitem{
	2018
} {
	3dimensions.kr. "3D design software that makes your design look like: Forté" (Translated from Korean)
}

\cvitem{
	2018
} {
	STAMPARE IN 3D. "Anthony Chen e lo strumento di disegno interattivo Forté"
}

\cvitem{
    2016
} {
	Branchemagasinet UDKOM. “3D-printere reparerer ting” 
}

\cvitem{
    2016
} {
	DIY 3D Printing. “Encore 3D Printing Upgrades for Everyday Objects”
}

\cvitem{
    2015
} {
	3dprint.com. “Sustainable 3D Printing Methods Add to or Subtract from Existing Objects”
}

\cvitem{
    2015
} {
	New Scientists. “3D print extra bits for old objects to help extend their life” 
}

\cvitem{
    2015
} {
	3ders.org. “Researchers develop Encore tool for augmenting everyday objects with 3D printing” 
}

\cvitem{
    2015
} {
	3dprint.com. “Encore: Research Allows for 3D Printed Augmentation of Everyday Objects”
}

\cvitem{
    2015
} {
	3dtectonix.com . “Encore Webgl-Based Tool and 3D Printing Improve Everyday Objects”
}

\cvitem{
    2014
} {
	labs.blogs.com. “Duet: Exploring Joint Interactions on a Smart Phone and a Smart Watch” 
}

\cvitem{
    2013
} {
	sourcebits.com. “How an Innovative Mobile Interaction Concept Could Benefit Enterprises”
}

\subsection{Collaborated research projects}

\cvitem{
	2018
}{
	\cvplace{
		Orecchio (collaborated with Xing-Dong Yang's group)
	}{
		EureAlert, Phys.Org, Dartmouth Press
	}	
}

\cvitem{
	2018
}{
	\cvplace{
		WrisText (collaborated with Xing-Dong Yang's group)
	}{
		Discovery's Daily Planet, QUARTZ, Weather Science, EureAlert
	}	
}


\cvitem{
	2018
}{
	\cvplace{
		Theromorph (collaborated with Lining Yao's group)
	}{
		CMU News, dezeen, ZDNet, ALL3DP
	}	
}


\cvitem{
	2016
}{
	\cvplace{
		SweepSense (collaborated with Gierad Laput)
	}{
		R\&D Magazine, MIT Technology Review
	}	
}

\cvitem{
	2016
}{
	\cvplace{
		Snap to It (collaborated with Adrian de Freitas)
	}{
		MIT Technology Review
	}	
}

\cvitem{
	2015
}{
	\cvplace{
		3D Printed Hair (collaborated with Gierad Laput)
	}{
		Fast Company, CNET, Gizmodo, Hackaday, MIT Technology Review, Engadget, Plastics Today, New York Magazine, etc.
	}	
}

\cvitem{
	2014
}{
	\cvplace{
		Skin Buttons (collaborated with Gierad Laput)
	}{
		New York Times, TechCrunch, WIRED, Fast Company, New Scientist, Gizmodo, CBC, etc.
	}	
}

\cvitem{
	2014
}{
	\cvplace{
		Tablet+Stylus Interaction (collaborated with Ken Hinckley)
	}{
		FastCo Design’s \#2 User Interface Innovation of 2014
	}	
}

\cvitem{
	2012
}{
	\cvplace{
		The Fat Thumb (collaborated with Sebastian Boring)
	}{
		PC World, Engadget, Gizmodo, etc.
	}	
}